\documentclass{article}
\usepackage{graphicx}

\title{Assignment No : A6}
\begin{document}
\maketitle

\noindent \textbf{\large Title :File Transfer using Socket Programming} \\[.75cm]
\textbf{\large Learning Objective : }\\	
\hspace*{5mm} 1. To understand about sockets	\newline
\hspace*{5mm} 2. To learn socket programming in python.	 \newline
\textbf{\large Problem Definition : }\\
\textbf{\large Related Mathematics :}\\
Let $S$ be the system such that\\
$$S=\lbrace s,e,X,Y,F_{me},F_{friend},DD,NDD |\phi_{s} \rbrace$$\\
where,\\
s = start state $S$\\
e = end state = file transferred\\
X = set of Inputs\\
Y = set of Outputs\\ 
\textbf{Input Analysis: }\\
$$X=\lbrace X1,X2 \rbrace$$\\
X1=input for server
X2=input for client
$$Y=\lbrace file \rbrace$$
where,\\
$$file=\lbrace A \cup N \cup S \rbrace$$
A=set of Alphabets \\
N=set of Numbers\\
S=set of symbols\\
$F_{me}$ is the main function where the execution of program begins i.e. the entry point of program.\\
$$F_{friend}=\lbrace f_1,f_2,f_3,f_4 \rbrace$$\\
where,\\
f_1 : be a function that creates server.\\
f_2 : client connects to server.\\
f_3 : that sends file\\
f_4 : receives file.
$x,y \in I^+ $\newpage
\textbf{\large Related Theory: }\\[.5cm]

\textbf{\large13.1.3 Sockets and Socket-based Communication :}
Sockets provide an interface for programming networks at the transport layer. Network communication
using Sockets is very much similar to performing fi le I/O. In fact, socket handle is treated like fi le handle.
The streams used in fi le I/O operation are also applicable to socket-based I/O. Socket-based
communication is independent of a programming language used for implementing it. That means, a socket
program written in Java language can communicate to a program written in non-Java (say C or C++) socket
program.
A server (program) runs on a specifi c computer and has a socket that is bound to a specifi c port. The
server listens to the socket for a client to make a connection request (see Fig. 13.4a). If everything goes
well, the server accepts the connection (see Fig. 13.4b). Upon acceptance, the server gets a new socket
bound to a different port. It needs a new socket (consequently a different port number) so that it can
continue to listen to the original socket for connection requests while serving the connected client.
 Socket\\[.5cm] 
\textbf{\large Socket Programming in Python. }\\[.5cm]
Python provides two levels of access to network services. At a low level, you can access the basic socket support in the underlying operating system, which allows you to implement clients and servers for both connection-oriented and connectionless protocols.

Python also has libraries that provide higher-level access to specific application-level network protocols, such as FTP, HTTP, and so on.
Server Socket Methods
Method	Description
s.bind()	This method binds address (hostname, port number pair) to socket.
s.listen()	This method sets up and start TCP listener.
s.accept()	This passively accept TCP client connection, waiting until connection arrives (blocking).

Client Socket Methods
Method	Description
s.connect()	This method actively initiates TCP server connection.

General Socket Methods
Method	Description
s.recv()	This method receives TCP message
s.send()	This method transmits TCP message
s.recvfrom()	This method receives UDP message
s.sendto()	This method transmits UDP message
s.close()	This method closes socket
socket.gethostname()	Returns the hostname.

\textbf{\large State Diagram: }\\
\begin{center}\includegraphics[scale=0.7]{state.jpeg}   \end{center}
where , \\
$q_0$=start state \\
$q_1$=run server \\
$q_2$=client establishes connecton with server \\
$q_3$=server sends file.\\
$q_4$=clients receives file.\\
$q_r$=connection not established.\\
$q_n$=end state\\[.75cm]
\textbf{\large Conclusion:}\\[.25cm]
In this assignment we have successfully implemented socket programming in python.we learnt basics\\
of socket programming.\\
\end{document}